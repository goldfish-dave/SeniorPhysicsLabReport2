\documentclass[aps,prl,twocolumn,groupedaddress,showkeys]{revtex4}

\usepackage{graphicx}

\begin{document}

\title{Thin Film Deposition}
\author{309248035 \\
				D.G. Wilcox}

\noaffiliation{}

\date{\today}

\begin{abstract}
Two techniques for thin film deposition, evaporation deposition and magnetic sputtering deposition, were used sequentially to get in-situ measurements of the resistivity of copper wrtit's thickness. Although it was hoped that the bulk resistivity of copper could be measured the number of measurements taken was estimated to be roughly half that required for observation. In contrast, the perculation threshold was observed at around 10 nm thickness.
\end{abstract}

\keywords{supttering deposition, evaporation deposition, bulk film resistivity, perculation threshold}


\maketitle


\section{Introduction}

The aim of this experiment was to measure the conductivity of a thin film of copper wrt thickness. The introduction deals with the concepts of evaporation and sputtering deposition techniques, as well as a technique for measuring thin film thickness. The procedure outlines how the required vacuum was obtained and conductivity was measured. The results are represented by a chart of Resistance Vs Thickness and is accompanied by some discussion, followed by conclusions.

\subsection{Thin Film Deposition Techniques}

Thin film deposition is a process that takes a source material and applies a thin film of it to a target substrate. The purposes are generally related to electronics, and involve thin films of conductors and the two techniques covered by this experiment are deposition via evaporation of the source material and deposition via sputtering of the source material.

\subsection{Evaporation Deposition}

The basic principle of Evaporation Deposition is to evaporate a metal (in this case aluminium) and allow it to condense on the target substrate. This forms an even coating of aluminium on the substrate. 

In more detail, the evaporation is done in a vacuum chamber. This is to avoid gas particles other than the aluminium being present and interfering with the process. The way the aluminium is boiled is by wrapping it around tungsten, which is then short circuited to provide heat. Since the melting point for tungsten is greater than the boiling point for aluminium, the tungsten is a suitable metal for the process.

\subsection{Sputtering Deposition}

Sputter deposition, rather then evaporating the source material, chips away from it causing the ejected particles to fall on the substrate material. This experiment used positively charged Argon ions from a plasma and accelerated them to the copper source, dislodging copper atoms and causing them to recondense on the substrate.


\subsection{Crystal Monitors}

In order to measure the thickness of the film while performing sputtering deposition a crystal monitor was used. A crystal monitor will measure the change in vibration frequency as it becomes more massive (due to the source material being deposited on its surface).


\section{Procedure}

\subsection{Preparation of the Substrate}

In order to perform evaporation deposition a vacuum of $1\times10^{-4}$ mbar was required. This was achieved by sequentially using a rotary pump and a diffusion pump.

Prior to evacuating the chamber the substrate (two glass microscope slides) was cleaned with alcohol and left on the floor of the chamber covered by a third slide as shown in figure~\ref{fig:slideArrangement}.

\begin{figure}[h]
	\includegraphics[width=0.6\linewidth]{glassSlideArrangement.png}
	\caption{Crossed glass slide arrangement\cite{labbook}}
	\label{fig:slideArrangement}
\end{figure}

Tungsten was wrapped in aluminium in the chamber in a way that we could provide about 30 A of current when short-ciruiting it.

Once everything was ready and in the chamber it was evacuated down to the required pressure. Once there the aluminium was evaporated until enough had condensed on the walls of the chamber to block light from passing through it. At this point the substrate was left to cool and then eventually removed.


\subsection{In-situ Conductivity Measurements}

For this section of the experiment a pressure of $1-6\times10^{-2}$ mbar was required (anywhere within the range was acceptable).

In the lid of the vacuum chamber was the magnetron source. This held the plasma in place while sputtering occurred.

One of the prepared substrates was held at a distance from the copper that was shared by the crystal monitor with the side that had been covered in a thin aluminium film facing towards the copper. The resistance of the substrate was measured via a multimeter connected from outside the chamber.

Once everything had been set up, and a flow of argon gas was provided that was in equilibrium with the vacuum pump at the required pressure, the plasma was created and sputtering occured. Using the crystal monitor to measure the thickness, the resistance of the substrate was recorded to a thickness of 25 nm.


\section{Results}

\begin{figure}[h]
	\includegraphics[width=\linewidth]{copperSputtering.png}
	\caption{Resistance Vs. Thickness}
	\label{fig:copperSputtering}
\end{figure}

\subsection{Uncertainties}

Although the plot shown in figure~\ref{fig:copperSputtering} has a smooth curve to it there are multiple possible causes for uncertainty in this experiment.

The first is the use of a crystal monitor. Although it provides accurate measurements to a couple of nanometres, it is possible for the monitor to expire. If the mass build up on the monitor becomes too great then the measurements provided will no longer be accurate.

Additionally, the monitor measures a vibrational change as a response of mass increase, not film thickness. By using a crystal monitor we are assuming that the film thickness is uniform, which is an unvalidated assumption.

Other causes for error include our measurement of the pressure in the chamber. Although throughout the process the pressure was indeed measured to be within our desired range, due to the size of the chamber it is unclear if that pressure was the actual pressure in the places where it was required.

Perhaps the most noteworthy cause of inaccuracy was the use of a plasma for the sputtering. Because the plasma is conductive measurements recorded while there was a plasma present where not measuring the resistance of the film on the substrate, but rather the plasma present in the chamber, which was wildly flucuating and not relevant to our experiment. In order to counter this all measurements of the resistance were after turning off the sputtering power for a period of time. Although this helped, the resistance was still observed to be in (greatly reduced) flucuation. This did not stop completely even after 30 seconds, but measurements were taken with only a 10 second pause because of time constraints. 

Finally, there was resistance due to the copper, but there was also resistance due to the aluminium. For accuracy the resistance of the aluminium would have been subtracted from the measured resistance, except for the fact that this resistance was so small compared to the flucuations observed as to be relatively insignificant.

\subsection{Discussion}

Figure~\ref{fig:copperSputtering} shows an undoubtable decrease in resistance with film thickness increase. Although the shown graph suggests that resistance will decrease consistently, tabulated values show that it will eventually bottom out. For further work it is suggested that measurements are taken at least until this bottoming out is observed.

Secondly it is clear that at around 10 nm the slope of the curve changes dramatically. This point is likely the perculation threshold. Up until then the copper did not form a contiguous layer ontop of the substrate but instead formed multiple adatoms on the surface. This would suggest that the notion of layer "thickness" prior to the perculation threshold is an inaccurate notion and an inconsistency between the crystal monitor (which was already covered in copper from previous experiments) and the substrate.

Finally figure~\ref{fig:copperResistivity} shows the resistivity Vs thickness. Unfortunately the curve does not flatten out and so it is not possible to compare our recorded value for the bulk resistivity of copper with tabulated values. That said, tabulated values give the resistivity of copper at $20^{\circ}$ C as $1.7\times10^{-8}$\cite{griffiths}. Thus in order to observe the bottoming out of the curve, assuming that it continues at the final observed gradient, measurements up to about 50 nm (at a rough estimate) would have to be recorded. Although this is feasible, it would certainly take a long time using the same experimental procedure. Automating the measurements and pressure regulation would be advised.

\begin{figure}[h]
	\includegraphics[width=\linewidth]{copperResistivity.png}
	\caption{Resistivity Vs. Thickness}
	\label{fig:copperResistivity}
\end{figure}

\section{Conclusions}

In this experiment the resistance of copper wrt to thickness was measured to a thickness of 25 nm. This was insufficient to observe the bulk resistance of copper but doubling the number of measurements was estimated to be sufficient to observe the bulk resistance. Since the measurements taken only took about half an hour, doubling the measurements would be feasible. However because of the fluctuating nature of the pressure and the time required to wait for the plasma to disperse, it is recomended that a way to automate the measurements or regulate the pressure be devised, in order to allow for more repeatable results.

Although the bulk resisitivity was not observed, the perculation threshold was. This is the threshold thickness at which the copper film becomes contiguous across the substrate and the rate of decrease of the resistivity decreases sharply. It was observed to be around 10 nm in thickness.


\bibliography{report}

\end{document}
